\section{Brokoli Sahne Soße}
  % Gerichttypen sind Vor, Haupt und Nach
  \label{Haupt:Brokoli_Sahne_Sosse}
    % Der Text hier kann dafür genutzt werden um dem Leser das
    % Gericht ein wenig schmackhaft zu machen und eine nette
    % Einleitung zu machen
    Wir kennen es alle. Nudeln mit roter Soße \dots wooohooo \dots, welch Event.
    Wenn die Lust auf eine Tomaten Soße gerade aber nicht so groß ist, aber der
    Hunger auf Nudeln aber dennoch da, kann ich diese Soße nur empfehlen. Eine
    schöne Käse-Sahne-Soße die es in sich hat ;-)

    \subsection*{Kurzinfo}
      % Es folgt eine sehr einfach Zusammenfassung über Dauer, Menge (für
      % wieviel Personen das Essen reicht) und Schwierigkeitsgrad
      \begin{itemize}
        \item Dauer:
          \begin{displaymath}
            % Nach dem kaufmännischen und-Zeichen kann die Zeit der jeweiligen
            % Aktion eingetragen werden.
            \begin{array}{l r}
              \text{Vorbereitung} & 15min\\
              \text{Ruhe} & 0min\\
              \text{Kochen} & 15min\\ \hline
              \text{gesamt} & 30min
            \end{array}
          \end{displaymath}
        % ANZAHL muss ersetzt werden für die Zahl, welche den Portionen
        % entspricht
        \item Für 3 Portionen
        % Da Schwierigkeitsgrade ohnehin schwer auf einen Nenner zu bringen
        % sind, reicht ein pi-mal-Daumen Prosaausdruck
        \item Schwierigkeit: absolut easy peasy
      \end{itemize}

    \subsection*{Zutaten}
      % Bei keinem Rezept sollte eine Angabe der benötigten Zutaten fehlen.
      % Typischwerweise werden Zutaten unterteilt, je nach dem zu welchem Teil
      % eines Gerichtes sie gehören. Zusätzliche Zutatengruppen können durch
      %
      % \item Zutatengruppe Y
      % \begin{enumerate}
      %   \item Zutat
      % \end{enumerate}
      %
      % hinzugefügt werden, wobei jede neue Zutat dur ein "\item Zutat"
      % innerhalb eines "begin-end enumerates" gekennzeichnet wird
      \begin{enumerate}
        \item Soße
        \begin{itemize}
          \item 150g Brokoli
          \item 2 Knoblauchzehen
          \item 100g Parmesankäse
          \item 200ml Sahne
          \item 2 Zwiebeln
        \end{itemize}
        \item Gewürze:
        \begin{itemize}
          \item Pfeffer
          \item Muskatnuss
          \item Salz
        \end{itemize}
      \end{enumerate}

    \subsection*{Zubereitung}
      % Als nächstes steht die Kochanleitung selbst an. Hierbei werden
      % zusammenhängende Schritte mit '\item' gekennzeichnet. Da es sich hier
      % immer noch um eine mehr als nur Hobby-projekt handelt, wäre ein bisschen
      % unterhaltsame Texte und kreativ geschriebenes natürlich sehr schön,
      % solange es verständlich bleibt ;-)
      \begin{enumerate}
        \item Da die Soße insgesamt nicht so dermaßen lange dauert kann man
              ruhig parallel schonmal die Nudeln kochen, sonst geht es wie
              üblich mit der Schnibbelei los. Hierfür die Zwiebel in mittel
              grobe Würfel schneiden und bei dem Brokoli die Röschen in kleine
              Mundgerechte Häbchen schneiden. Wenn man einmal an dem Stiel des
              Brokolis langschneidet, kann man auch den Würfeln. Bei dem Knoblauch,
              sofern Knoblauchpresse vorhanden, nur eben die Haut abziehen,
              sonst zusätzlich auch noch würfeln.
        \item Für ein bisschen schmackofatz Röstaroma, den Brokoli in einer
              heißen Pfanne etwas anbrutzeln. Wenn der Brokoli hier und da ein
              paar angeknusperte Stellen hat kann dieser wieder raus.
        \item Als nächstes Zwiebeln und Knoblauch in einer Pfanne lecker
              brutzeln. Noch bevor irgendwelche Gewürze mit in die Pfanne
              kommen, wird ein zirka die Hälfte des Parmesans in die Pfanne.
        \item Bevor der Parmesan anbrennt, kommt der Brokoli wieder in die
              Pfanne und ein weiteres Viertel des Parmesans in die Pfanne. Da
              der Parmesan schon flüßig und flexibel ist, hat er die Chance sich
              in den Brokoliröschen breit zu machen, das ist der Hammer.
        \item Nach weiteren zwei Minuten kommen die Sahne hinzu, bevor in der
              Pfanne alles hart wird. Das Gebräu ordentlich umrühren. Es könnte
              sein, dass am Boden der Pfanne manche Dinge festsitzen, aber mit
              der Flüßigkeit aus der Sahne sollten sie sich lösen lassen. Sollte
              die Sahne nicht sähmig werden, rein mit dem guten Zeug. Da kann
              dann wohl noch ein bisschen Parmesan rein. Zu viel Käse hat noch
              nie geschaded :D
      \end{enumerate}
