\section{Vegetarische Bulletten}
  \label{Haupt:Vegetarische_Bulletten}
  Falls jemand Bock auf Burger oder Dergleichen, aber aus welchen Gründen auch
  immer derzeit auf Fleisch verzichtet, gibt es einen alternativen Belag für
  Burger und co.

  \subsection*{Kurzinfo}
    \begin{itemize}
      \item Dauer
        \begin{displaymath}
          \begin{array}{l r}
            \text{Vorbereitung}& 10min \\
            \text{Ruhe}& 2h\\
            \text{Kochen}& 10 min\\ \hline
            \text{gesamt}& 2h20min
          \end{array}
        \end{displaymath}
      \item Für 4 Portionen
      \item Schwierigkeits: absolut easy peasy
    \end{itemize}
  \subsection*{Zutaten}
    \begin{enumerate} 
      \item Bulletten-Pampe:
      \begin{itemize}
        \item 1-2 Eier
        \item 100g Körnigen Frischkäse
        \item 150g kernige Haferflocken
        \item Öl
        \item 1 Teelöffel Senf
        \item 1 Zwiebel
      \end{itemize}
      \item Gewürze:
      \begin{itemize}
        \item Korriander
        \item Paprika (edelsüß \& scharf)
        \item Gehackte Petersilie
        \item Pfeffer
        \item Salz
      \end{itemize}
    \end{enumerate}

  \subsection*{Zubereitung}
    \begin{enumerate}
      \item Die Zwiebel in kleine Würfel schneiden. Wenn man etwas verrückt ist,
            kann man die Zwiebelwürfelchen ein bisschen anbraten, bevor man die
            ganze Pampe zusammen schmiert.
      \item Haferflocken, und Zwiebeln in eine kleine Schüssel tun und ein
            bisschen durch shakern und großzügig würzen. Die Schüssel solange
            durchrühren, bis die Zwiebeln keine weiteren Gewürze mehr aufnehmen
            können. Da die Zwiebelwürfel schön klein sind, verteilen sich die
            gewürde somit auch gut in der Masse.
      \item Als nächstes den Frischkäse und den Senf hinzufügen und wieder gut
            durchmischen. Die Masse wird deutlich klebriger, da muss man
            vielleicht dann vielleicht einfach mal mit sein kleinen Fingers rein
            ;-)
      \item Wenn die Haferflocken und der körnige Frischekäse gut vermischt
            wird zunächst ein Ei darin aufgeschlagen und gut mit der Masse
            vermischt. Sollte die Konsistenz immer noch der \emph{bröselig}
            sein, muss noch ein zweites Ei dran glauben und untergemischt
            werden.
      \item Wenn alles gut untergemischt ist, kann die Schüssel für ein paar
            Stunden in den Kühlschrank, damit alles Gelegenheit hat schön
            durchzuziehen.
      \item Eine Pfanne mit etwas Öl auf mittlere Temperatur bringen.
            Während die Pfanne aufwärmt kann man bereits Bulletten formen.
            Dafür empfpiehlt es sich die
            Hände immer mal wieder ein bisschen anzufeuchten, da die Paste sonst
            doch sehr an der Hand kleben bleibt und ein vernünftiges formen sehr
            erschwert.
      \item Bulletten in die Pfanne schmeißen. Ein geeigneter Punkt zum Wenden
            ist, wenn die helle Farbe auf der oberen Seite der Matsche langsam
            dunkler wird.
      \item Wenn die Bullette auf beiden Seiten gebrutzelt wurde, kann sie
            einfach verputzt werden =)
    \end{enumerate}

  \subsection*{Variationsideen}
    \begin{itemize}
      \item Falls man mal Lust auf etwas orientalischeres hat, als Burger, lässt
            sich dies Rezept auch zu einer pseudo-Fallaffel erweitern. Dafür
            kann man dann einfach den Senf weg lassen und an Stelle dessen
            Gewürze wie Curry und Kurkuma ein mischen.
    \end{itemize}
