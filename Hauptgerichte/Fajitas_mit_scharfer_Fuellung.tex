\section{Fajitas mit scharfer Füllung}
  \label{Haupt:Fajitas_mit_scharfer_Fuellung}
  Ok Freunde der Nacht, diese Dinge sind ein paar geile gefüllte platte Fladen
  gefüllt mir lecker scharfen Gemüse und Zeug. Ganz ehrlich, damit kann man auch
  richtig hungrige Mäuler stopfen und es könnte sogar passieren, dass es dabei
  noch fast gesund ist.
  \subsection*{Kurzinfo}
    \begin{itemize}
      \item Dauer
        \begin{displaymath}
          \begin{array}{l r}
            \text{Vorbereitung}& 15min \\
            \text{Kochen}& +15min\\ \hline
            \text{gesamt}& 30min
          \end{array}
        \end{displaymath}
      \item Für 3 Portionen
      \item Schwierigkeits: absolut easy peasy
    \end{itemize}
  \subsection*{Zutaten}
    \begin{enumerate} 
      \item Füllung:
      \begin{itemize}
        \item 200g Brokoli
        \item 4 mittel große Champinions
        \item 150g Möhren
        \item Öl
        \item 1 Paprika
        \item 1 Zuccini
        \item 1 Zwiebel
      \end{itemize}
      \item Gewürze:
      \begin{itemize}
        \item Chilliflocken
        \item Curry
        \item Honig (oder Zucker)
        \item Paprikagewürz (edelsüß \& scharf)
        \item Pfeffer
        \item Salz
      \end{itemize}
      \item Fajita:
      \begin{itemize}
        \item 3 dieser Fladendinger
        \item 100g Joghurt
        \item 100g geriebenen Käse
      \end{itemize}
    \end{enumerate}

  \subsection*{Zubereitung}
    \begin{enumerate}
      \item Zunächst muss erstmal all das Gemüse gewaschen und geschnibbelt
            werden, ist doch klar. Formen sind da eigentlich ziemlich Wumpe,
            allerdings sei gesagt, je sperriger die Formen sind, desto mehr
            Luft ist nachher in der Tasche ;-)
      \item Ein bisschen Öl in der Pfanne heiß werden lassen, und die
            geschnibbelten Möhren und Zwiebel mit einer Priese Chilliflocken,
            den beiden Paprikagewürzen und Curry anbrutzeln, das kann ruhig max
            power sein.
      \item Wenn die Zwiebels langsam glasig werden kann der geschnibbelte Rest 
            in die Pfanne geschubbst werden. Um ein bisschen geiles Röstaroma zu
            bekommen, zwischen durch auch einen Moment den Klüngel in der Pfanne
            ruhig nicht durchmischen.
      \item Je nachdem wie intensiv die erste Runde Gewürze war, empfiehlt es sich
            jetzt nochmal den Finger in die Pfanne zu stecken und abzuschmecken.
      \item Naja, vorm verputzen, muss nur noch der der Fladen mit ein bisschen
            Joghurt bepinselt werden, nach Wunsch mit Käse bestreuen und sich
            den Wanzt voll schlagen =)
    \end{enumerate}

  \subsection*{Variationsideen}
    \begin{itemize}
      \item Wer Bock hat, kann sich das Ganze zu einem Chilli-im-Fladen machen,
            in dem noch ein bisschen Kidneybohnen und Mais den Weg in die Pfanne
            finden. Dazu noch eine gewürfelte Tomate und fertig ist der
            Gaumenschmaus.
      \item Für die fleischhungrigen unter uns, kann man sich auch gewünschtes
            Fleisch wie Hänchen und co. in Scheiben schneiden und ebenfalls
            scharf anbrutzeln. Falls man aus welchen Gründen auch immer kein
            Fleisch isst, kann man sich auch
            \nameref{Haupt:Vegetarische_Bulletten} mit einer etwas
            orientalischeren Gewürzmischung in kleine Kügelchen formen, anbraten
            und ebenfalls im Fladen versenken.
    \end{itemize}
