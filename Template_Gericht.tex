\section{GERICHTNAME}
  % Gerichttypen sind Vor, Haupt und Nach
  \label{GERICHTTYPE:GERICHTNAME}
    
    % Der Text hier kann dafür genutzt werden um dem Leser das
    % Gericht ein wenig schmackhaft zu machen und eine nette
    % Einleitung zu machen

    \subsection*{Kurzinfo}
      % Es folgt eine sehr einfach Zusammenfassung über Dauer, Menge (für
      % wieviel Personen das Essen reicht) und Schwierigkeitsgrad
      \begin{itemize}
        \item Dauer:
          \begin{displaymath}
            % Nach dem kaufmännischen und-Zeichen kann die Zeit der jeweiligen
            % Aktion eingetragen werden.
            \begin{array}{l r}
              \text{Vorbereitung} & \\
              \text{Ruhe} & \\
              \text{Kochen} & \\ \hline
              \text{gesamt} &
            \end{array}
          \end{displaymath}
        % ANZAHL muss ersetzt werden für die Zahl, welche den Portionen
        % entspricht
        \item Für ANZAHL Portionen
        % Da Schwierigkeitsgrade ohnehin schwer auf einen Nenner zu bringen
        % sind, reicht ein pi-mal-Daumen Prosaausdruck
        \item Schwierigkeit
      \end{itemize}

    \subsection*{Zutaten}
      % Bei keinem Rezept sollte eine Angabe der benötigten Zutaten fehlen.
      % Typischwerweise werden Zutaten unterteilt, je nach dem zu welchem Teil
      % eines Gerichtes sie gehören. Zusätzliche Zutatengruppen können durch
      %
      % \item Zutatengruppe Y
      % \begin{enumerate}
      %   \item Zutat
      % \end{enumerate}
      %
      % hinzugefügt werden, wobei jede neue Zutat dur ein "\item Zutat"
      % innerhalb eines "begin-end enumerates" gekennzeichnet wird
      \begin{enumerate}
        \item Irgendeine Zutatengruppe X
        \begin{itemize}
          \item Zutat für Zutatengruppe X
        \end{itemize}
        \item Gewürze:
        \begin{itemize}
          \item irgendein Gewürz
          \item ein anderes Gewürz
        \end{itemize}
      \end{enumerate}

    \subsection*{Zubereitung}
      % Als nächstes steht die Kochanleitung selbst an. Hierbei werden
      % zusammenhängende Schritte mit '\item' gekennzeichnet. Da es sich hier
      % immer noch um eine mehr als nur Hobby-projekt handelt, wäre ein bisschen
      % unterhaltsame Texte und kreativ geschriebenes natürlich sehr schön,
      % solange es verständlich bleibt ;-)
      \begin{enumerate}
        \item Schritt 1 ...
        \item Schritt 2 ...
      \end{enumerate}

    \subsection*{Variationsideen}
      % Falls jemand Ideen hat wie ein Gericht ein wenig abgewandelt oder
      % erweitert werden kann, so kann sich die Person in dieser Sektion
      % auslassen. Hierbei gehört eine Idee jeweils hinter ein "\item"
      \begin{itemize}
        \item Idee 1
      \end{itemize}
