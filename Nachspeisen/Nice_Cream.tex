\section{Nice Cream}
  % Gerichttypen sind Vor, Haupt und Nach
  \label{GERICHTTYPE:GERICHTNAME}
    
    % Der Text hier kann dafür genutzt werden um dem Leser das
    % Gericht ein wenig schmackhaft zu machen und eine nette
    % Einleitung zu machen
    Wir kennen das alle, es is Sommer und man sehnt sich nach einer kleinen
    Abkühlung. Problematik: Nicht immer hat man ein kleines Plantschbecken zur
    Hand. Aber, damit ist jetzt Schluss. Der offizielle Plantschbeckenersatz
    nennt sich laut Internet \emph{Nice-Cream} und hat soweit überhaupt gar
    nichts mit einem Plantschbecken am Hut, sondern ist eine vegane Eis
    Alternative. Leider ist die Zubereitung so einfach, dass ich mich fast nicht
    traue dieses Rezept hier aufzulisten. Naja, Arsch oder König!

    \subsection*{Kurzinfo}
      % Es folgt eine sehr einfach Zusammenfassung über Dauer, Menge (für
      % wieviel Personen das Essen reicht) und Schwierigkeitsgrad
      \begin{itemize}
        \item Dauer:
          \begin{displaymath}
            % Nach dem kaufmännischen und-Zeichen kann die Zeit der jeweiligen
            % Aktion eingetragen werden.
            \begin{array}{l r}
              \text{Vorbereitung} & 2min\\
              \text{Ruhe} & 6h\\
              \text{Kochen} & 2min\\ \hline
              \text{gesamt} &6h4min
            \end{array}
          \end{displaymath}
        % ANZAHL muss ersetzt werden für die Zahl, welche den Portionen
        % entspricht
        \item Für 1 Portionen
        % Da Schwierigkeitsgrade ohnehin schwer auf einen Nenner zu bringen
        % sind, reicht ein pi-mal-Daumen Prosaausdruck
        \item Schwierigkeit: absolut easy peasy
      \end{itemize}

    \subsection*{Zutaten}
      % Bei keinem Rezept sollte eine Angabe der benötigten Zutaten fehlen.
      % Typischwerweise werden Zutaten unterteilt, je nach dem zu welchem Teil
      % eines Gerichtes sie gehören. Zusätzliche Zutatengruppen können durch
      %
      % \item Zutatengruppe Y
      % \begin{enumerate}
      %   \item Zutat
      % \end{enumerate}
      %
      % hinzugefügt werden, wobei jede neue Zutat dur ein "\item Zutat"
      % innerhalb eines "begin-end enumerates" gekennzeichnet wird
      \begin{enumerate}
        \item Creme:
        \begin{itemize}
          \item 1 braune Bananen
          \item 3 Esslöffel Backkakao
        \end{itemize}
      \end{enumerate}

    \subsection*{Zubereitung}
      % Als nächstes steht die Kochanleitung selbst an. Hierbei werden
      % zusammenhängende Schritte mit '\item' gekennzeichnet. Da es sich hier
      % immer noch um eine mehr als nur Hobby-projekt handelt, wäre ein bisschen
      % unterhaltsame Texte und kreativ geschriebenes natürlich sehr schön,
      % solange es verständlich bleibt ;-)
      \begin{enumerate}
        \item Die braune Banane schälen und in kleine Stückchen schneiden. Die
              Stückchen in eine Tuperdose oder Gefrierbeutel packen und ab ins
              Eisfach damit, soll ja schließlich erfrischen.
        \item Wenn du Bananenstückchen gefroren sind, werden sie ganz locker
              aufgeschlagen. Ich empfehle irgendeine Art von Zerhacker oder
              Zauberstab, ein einfacher Mixer könnte damit Probleme bekommen.
              Sobald die Bananen aus dem Kühlschrank sind, läuft übrigens die
              Zeit. Wer zu lange mit dem \emph{schaumig aufschlagen} beschäftigt
              ist, hat nachher mehr Suppe als Creme.
        \item Wenn das Mus eine sähmige Konsitenz hat, brauch man nur noch den
              Kakao dazu mischen. Da es offensichtlich eine Geschmacksfrage ist
              wie intensiv es nach Kakao schmecken soll, habe ich mit den 3
              Esslöffel in der Zutatenliste mal nach oben abgeschätzt. Naja,
              das ist es dann aber auch schon.
      \end{enumerate}

    \subsection*{Variationsideen}
      % Falls jemand Ideen hat wie ein Gericht ein wenig abgewandelt oder
      % erweitert werden kann, so kann sich die Person in dieser Sektion
      % auslassen. Hierbei gehört eine Idee jeweils hinter ein "\item"
      \begin{itemize}
        \item Variation ist hierbei gar nicht so leicht, da Banane der Creme die
              Geschmeidigkeit gibt. Die Frage ist also, womit kann man Banane
              mischen, das es immer noch ein fancy Geschmack hat, eine Idee wäre
              gefrorene Himbeere.
      \end{itemize}
